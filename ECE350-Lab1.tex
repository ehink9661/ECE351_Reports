\documentclass{article}
\usepackage[utf8]{inputenc}

%%%%%%%%%%%%%%%%%%%%%%%%%%%%%%%%%%%%%%%%%%%%%%%%%%%%%%%%%%%%%%%%
%                                                              %
% Ethan Hinkle                                                 %
% ECE 351-51                                                   %
% Lab 1                                                        %
% January 18th, 2022                                           %
% This lab is meant to teach us about Spyder, Python in        %
% general, and using Latex. The summaries of each part can be  %
% found in their corresponding sections below.                 %
%                                                              %
%%%%%%%%%%%%%%%%%%%%%%%%%%%%%%%%%%%%%%%%%%%%%%%%%%%%%%%%%%%%%%%%

\title{Ethan Hinkle - ECE 351 - Lab 1}
\author{Ethan Hinkle}
\date{January 18th, 2022}

\begin{document}

\maketitle

\section{Part 1}
The purpose of part 1 was to get familiar with the Spyder IDE and using Python 3.x. This was accomplished by viewing the built in tutorial for Spyder to get more acquainted with the systems. Then I pulled up the SpyderKeyboardShortcutsEditor pdf off of Canvas and reviewed the keyboard shortcuts for Spyder that were listed there (the shift+tab command was one I had never seen before but seemed useful for longer programs). Once this review was done I created a new python file in spyder and named it Hinkle\_Ethan\_ECE351\_Lab1.py.
\section{Part 2}
The purpose of part 2 was to introduce us to useful methods, operations, syntax, commands, and debugging tools in python. The first part of this lesson was over variables, and that their type does not need to be declared in python and arithmetic operations. Then we went over how to use lists and arrays in python and how they operate. We then reviewed how \# can be used for comments before seeing more ways to make arrays and navigate their contents. We then saw how to create plots in python and their behavior changes depending on how they were coded. We then went over how to use imaginary numbers in our code and how to properly format them for success. Finally we reviewed python packages and commands that we will use during the semester for review.
\section{Part 3}
Part 3 went over pep8 coding practices that we will be using during this lab. The first one was to avoid using tab for indentations, and instead just use 4 spaces to avoid problems. Then we learned to use docstrings to define functions or programs we create. Next we learned how to wrap lines so that we never exceed 79 characters in a single line for readability, setting a visual reminder in our preferences. Next we reviewed proper commenting etiquette, with  block comments forming complete sentences with periods, while shorter identifying comments do not require this. Next we learned to put spaces around operators and after commas, but not inside brackets to improve readability and cut down on wasted space. Finally, we went over the many naming conventions we will be using in our code and where they are usually used in code (lower\_case\_with\_underscores for funtions and methods, CapWords for classes). I was taught to camel case (or mixedCase) my code so will need to practice these conventions.
\section{Part 4}
Part 4 breaks away from learning python and breaks into learning Latex. The lesson was divided into three parts. the first part had us read the cheat sheet given for Latex, studying how to write mathematical expressions using \$, how to create new lines, write comments, make lists, and insert special symbols. The second part dealt with finding a template to use for the lab, or creating your own. The final part informed us that we can also insert code, images, and complex math equations into the report.
\section{Questions}
1. Which course are you most excited for in your degree? Which course have you enjoyed the most so far?\\
I would say that the course I am most exited for in Computer Engineering is my current Digital Systems Engineering (ECE 440) class. I am interested in working with Verilog and FPGAs to better prepare myself for industry so feel that I will enjoy the class, even if I am not a huge fan of how Dr J structures the class. So far the most enjoyable course for me would be ECE 212, as Professor Frenzel made the class enjoyable and informative without burying me in work.\\
\\
2. Leave any feedback on the clarity of the expectations, instructions, and deliverables.\\
When we decide on a template do you want us to stick to it? I only gave brief summaries about what each part of this lab went over as that is what I thought was wanted, no in depth explanation about what I learned or thought was cool. I didn't use the template for this document as it was already created when I got to that part of the lab.
\end{document}
